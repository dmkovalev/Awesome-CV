% rubber: set program xelatex
%!TEX TS-program = xelatex
%!TEX encoding = UTF-8 Unicode
% Awesome CV LaTeX Template
%
% This template has been downloaded from:
% https://github.com/posquit0/Awesome-CV
%
% Authors:
% Claud D. Park <posquit0.bj@gmail.com>
% Lars Richter <mail@ayeks.de>
%
% Template license:
% CC BY-SA 4.0 (https://creativecommons.org/licenses/by-sa/4.0/)
%


%%%%%%%%%%%%%%%%%%%%%%%%%%%%%%%%%%%%%%
%     Configuration
%%%%%%%%%%%%%%%%%%%%%%%%%%%%%%%%%%%%%%
%%% Themes: Awesome-CV
\documentclass[11pt, a4paper]{awesome-cv}

%%% Override a directory location for fonts(default: 'fonts/')
\fontdir[fonts/]

%%% Configure a directory location for sections
%\newcommand*{\sectiondir}{cv/}

%%% Override color
% Awesome Colors: awesome-emerald, awesome-skyblue, awesome-red, awesome-pink, awesome-orange
%                 awesome-nephritis, awesome-concrete, awesome-darknight
%% Color for highlight
% Define your custom color if you don't like awesome colors
\colorlet{awesome}{awesome-red}
%\definecolor{awesome}{HTML}{CA63A8}
%% Colors for text
%\definecolor{darktext}{HTML}{414141}
%\definecolor{text}{HTML}{414141}
%\definecolor{graytext}{HTML}{414141}
%\definecolor{lighttext}{HTML}{414141}

%%% Override a separator for social informations in header(default: ' | ')
%\headersocialsep[\quad\textbar\quad]


%%%%%%%%%%%%%%%%%%%%%%%%%%%%%%%%%%%%%%
%     Personal Data
%%%%%%%%%%%%%%%%%%%%%%%%%%%%%%%%%%%%%%
%%% Essentials
\name{Dmitry }{Kovalev}
\address{119590, Mosfilmovskaya 70-3-552, Moscow, Russia}
\mobile{(+7) 916-153-0375} 
%%% Social
\email{dm.kovalev@gmail.com}
\homepage{synthesis.ipi.ac.ru/synthesis/staff/dmkovalev}
\github{dmkovalev}
\linkedin{dmykovalev}
%%% Optionals
\position{Research Scientist{\enskip\cdotp\enskip}Teaching Assistant}


%%%%%%%%%%%%%%%%%%%%%%%%%%%%%%%%%%%%%%
%     Letter Data (Cover Letter)
%%%%%%%%%%%%%%%%%%%%%%%%%%%%%%%%%%%%%%
\recipient
  {Dr. Michael L. Brodie}
  {CSAIL, MIT}
\letterdate{\today}
\lettertitle{Visiting Research Possibility}
\letteropening{Dear Dr. Brodie,}
\letterclosing{Sincerely,}
\letterenclosure[Attached]{Curriculum Vitae}


%%%%%%%%%%%%%%%%%%%%%%%%%%%%%%%%%%%%%%
%     Content
%%%%%%%%%%%%%%%%%%%%%%%%%%%%%%%%%%%%%%
%%% Make a footer for CV with three arguments(<left>, <center>, <right>)
\makecvfooter
  {\today}
  {Dmitry Kovalev~~~·~~~Cover Letter}
  {}

\begin{document}
%%% Make a header for CV using personal data
\makecvheader

%%% Make a title for Cover Letter using letter data
\makelettertitle

%%% Write content for your cover letter
\begin{cvletter}


\cvparagraph{About Me}
I am writing this letter to ask if there is a possibility to join your research group. I am deeply interested in 



Opening paragraph: State why you are writing; how you learned of the organization or position, and basic information about yourself.

\cvparagraph{Why Google?}
2nd paragraph: Tell why you are interested in the employer or type of work the employer does (Simply stating that you are interested does not tell why, and can sound like a form letter). Demonstrate that you know enough about the employer or position to relate your background to the employer or position. Mention specific qualifications which make you a good fit for the employer’s needs. (Focus on what you can do for the employer, not what the employer can do for you.) This is an opportunity to explain in more detail relevant items in your resume. Refer to the fact that your resume is enclosed. Mention other enclosures if such are required to apply for a position.

\cvparagraph{Why Me?}
3rd paragraph: Indicate that you would like the opportunity to interview for a position or to talk with the employer to learn more about their opportunities or hiring plans. State what you will do to follow up, such as telephone the employer within two weeks. If you will be in the employer’s location and could offer to schedule a visit, indicate when. State that you would be glad to provide the employer with any additional information needed. Thank the employer for her/his consideration.

\end{cvletter}

%%% Make a closing for Cover Letter using letter data
\makeletterclosing

\end{document}
